% ------------------------------------------------------------------
\documentclass[12 pt]{article} % A4 paper set by geometry package below
\pagenumbering{arabic}
\setlength{\parindent}{10 mm}
\setlength{\parskip}{12 pt}

% Nimbus Sans font should be reasonably legible
\usepackage{helvet}
\renewcommand{\familydefault}{\sfdefault}
\usepackage[T1]{fontenc}  % Without this \textsterling produces $

% Section header spacing
\usepackage{titlesec}
\titlespacing\section{0pt}{12pt plus 4pt minus 2pt}{0pt plus 2pt minus 2pt}
\titlespacing\subsection{0pt}{12pt plus 4pt minus 2pt}{0pt plus 2pt minus 2pt}
\titlespacing\subsubsection{0pt}{12pt plus 4pt minus 2pt}{0pt plus 2pt minus 2pt}

\usepackage{amsmath}
\usepackage{amssymb}
\usepackage{graphicx}
\usepackage{verbatim}    % For comment
\usepackage[paper=a4paper, marginparwidth=0 cm, marginparsep=0 cm, top=2.5 cm, bottom=2.5 cm, left=3 cm, right=3 cm, includemp]{geometry}
\usepackage[pdftex, pdfstartview={FitH}, pdfnewwindow=true, colorlinks=true, citecolor=blue, filecolor=blue, linkcolor=blue, urlcolor=blue, pdfpagemode=UseNone]{hyperref}

% Put module code and last-modified date in footer
\usepackage{fancyhdr}
\pagestyle{fancy}
\fancyhf{}
\renewcommand{\headrulewidth}{0pt}
\cfoot{{\small \thisunit}\hfill \thepage\hfill {\small \moddate}}

% Hopefully address Canvas complaints about pdf tagging
%\usepackage[tagged]{accessibility}
\hypersetup {
  pdfauthor={David Schaich},
  pdftitle={Statistical Physics Tutorial Activity},
}
% ------------------------------------------------------------------



% ------------------------------------------------------------------
% Shortcuts
\newcommand{\be}{\ensuremath{\beta} }
\newcommand{\eps}{\ensuremath{\varepsilon} }
\newcommand{\om}{\ensuremath{\omega} }
\newcommand{\vev}[1]{\ensuremath{\left\langle #1 \right\rangle} }
\newcommand{\pderiv}[2]{\ensuremath{\frac{\partial #1}{\partial #2}} }
% ------------------------------------------------------------------



% ------------------------------------------------------------------
\begin{document}
\newcommand{\thisunit}{MATH327 Tutorial (Solid)}
\newcommand{\moddate}{Last modified 21 Mar.~2024}
\begin{center}
  {\Large \textbf{MATH327: Statistical Physics, Spring 2024}} \\[12 pt]
  {\Large \textbf{Tutorial activity \ --- \ Einstein solid}} \\[24 pt]
\end{center}

This activity will be introduced in our 21 March tutorial, which happens to be the last one before an extended break.
While you're welcome to work on the tasks below over the break, we will re-introduce this activity when we meet again in April, and you will have until week 10 (25 April) to continue working on it.
The tasks below all involve the micro-canonical and canonical ensembles, so you already have everything you need to solve them.

As a motivational warm-up exercise, consider the $N$ distinguishable spins in a solid we analyzed in Section~3.4 of the lecture notes.
We found that this system has internal energy
\begin{equation*}
  E = -NH\tanh(\be H)
\end{equation*}
for inverse temperature $\be = 1 / T$ and magnetic field strength $H$.
What is the corresponding heat capacity?
How does it compare to the experimental\footnote{Experimentally it is easier to measure the heat capacity at constant \textit{pressure}, $c_p$, rather than at constant volume, but the difference between $c_p$ and $c_v$ is negligible for our purposes here.} data points in the figure below (from Schroeder's \textit{Introduction to Thermal Physics})?

\begin{center}\includegraphics[width=0.9\textwidth]{figs/heat_cap.pdf}\end{center}

You should find poor agreement --- especially upon turning off the external field by taking $H \to 0$!
This issue turns out to persist even for more realistic models of solids analyzed using classical approaches.
To address it, in 1907 Einstein developed a simple model of solids based on quantized energies, taking some inspiration from his famous 1905 proposal that quantized energies explain the \href{https://en.wikipedia.org/wiki/Photoelectric_effect}{photoelectric effect}.

\newpage % TODO: Spacing hack...
The `Einstein solid' consists of many atoms whose positions are fixed to (distinguishable) locations in a regular lattice.
Interactions between neighbouring atoms are credited with pinning each atom to its fixed location.
This is modeled by picturing neighbouring atoms connected by `oscillators', analogous to springs, which possess energy as a consequence of these interactions.
We define the Einstein solid by hypothesizing that the energy of each oscillator is quantized, $\eps_i = 0, \hbar \om$, $2\hbar \om$, $\cdots$, with the same characteristic angular frequency \om for all oscillators.
Although these oscillators model interactions between nearest-neighbour atoms, in this approach they are \textit{non-interacting} degrees of freedom that we can analyze using the statistical physics tools we have already developed.

As illustrated by the figure below, also from Schroeder's \textit{Introduction to Thermal Physics}, the number of oscillators depends both on the number of atoms and their layout.
In this two-dimensional square lattice, $N$ oscillators would correspond to $N / 2$ atoms in the solid.
In a three-dimensional cubic lattice, $N$ oscillators would correspond to $N / 3$ atoms. \\[-24 pt]
\begin{center}\includegraphics[width=0.5\textwidth]{figs/solid.pdf}\end{center}

Our goal is to compute the heat capacity for an Einstein solid.
Let's begin by working in terms of the micro-canonical ensemble, fixing the total energy
\begin{equation*}
  E = \sum_i \eps_i = \sum_i k_i\hbar\om \equiv K\hbar\om
\end{equation*}
where $K \equiv \sum_i k_i$ is the integer number of energy `units' available to be distributed among the $N$ oscillators.
Each different way of distributing these $K$ units of energy among the $N$ (distinguishable) oscillators defines a unique micro-state.

\begin{itemize}
  \item What is the total number of micro-states in terms of $N$ and $K$?
        Check your result for a minimal three-oscillator system when it has $K = 0$, $1$, $2$ or $3$ units of energy.

  \item Now consider $K \gg 1$ and $N \gg 1$, so that we can apply Stirling's formula and also approximate $N - 1 \approx N$ and $K - 1 \approx K$.
        What is the resulting entropy?

  \item What is the temperature $T(E)$ of the Einstein solid?
        Is it a `natural' system with non-negative temperature?
\end{itemize}

Now we will change perspective to work in terms of the canonical ensemble, by inverting $T(E)$ to find the internal energy expectation value in terms of the temperature.
Differentiating this $\vev{E}\!(T)$ will then provide the heat capacity $c_v$ for the Einstein solid.
What is $c_v$ in terms of $x \equiv \hbar \om / T$?
How does it compare to the experimental data points shown above?
In particular, what are the leading corrections to the high- and low-temperature limits of $c_v$?

While the Einstein solid describes the experimental data much better than the non-interacting spins we first considered, there is still room for improvement\dots

\end{document}
% ------------------------------------------------------------------
