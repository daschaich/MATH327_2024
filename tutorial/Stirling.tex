% ------------------------------------------------------------------
\documentclass[12 pt]{article} % A4 paper set by geometry package below
\pagenumbering{arabic}
\setlength{\parindent}{10 mm}
\setlength{\parskip}{12 pt}

% Nimbus Sans font should be reasonably legible
\usepackage{helvet}
\renewcommand{\familydefault}{\sfdefault}
\usepackage[T1]{fontenc}  % Without this \textsterling produces $

% Section header spacing
\usepackage{titlesec}
\titlespacing\section{0pt}{12pt plus 4pt minus 2pt}{0pt plus 2pt minus 2pt}
\titlespacing\subsection{0pt}{12pt plus 4pt minus 2pt}{0pt plus 2pt minus 2pt}
\titlespacing\subsubsection{0pt}{12pt plus 4pt minus 2pt}{0pt plus 2pt minus 2pt}

\usepackage{amsmath}
\usepackage{amssymb}
\usepackage{graphicx}
\usepackage{verbatim}    % For comment
\usepackage[paper=a4paper, marginparwidth=0 cm, marginparsep=0 cm, top=2.5 cm, bottom=2.5 cm, left=3 cm, right=3 cm, includemp]{geometry}
\usepackage[pdftex, pdfstartview={FitH}, pdfnewwindow=true, colorlinks=true, citecolor=blue, filecolor=blue, linkcolor=blue, urlcolor=blue, pdfpagemode=UseNone]{hyperref}

% Put module code and last-modified date in footer
\usepackage{fancyhdr}
\pagestyle{fancy}
\fancyhf{}
\renewcommand{\headrulewidth}{0pt}
\cfoot{{\small \thisunit}\hfill \thepage\hfill {\small \moddate}}

% Hopefully address Canvas complaints about pdf tagging
%\usepackage[tagged]{accessibility}
\hypersetup {
  pdfauthor={David Schaich},
  pdftitle={Statistical Physics Tutorial Activity},
}
% ------------------------------------------------------------------



% ------------------------------------------------------------------
% Shortcuts
\newcommand{\cO}{\ensuremath{\mathcal O} }
\newcommand{\eps}{\ensuremath{\epsilon} }
\newcommand{\Ga}{\ensuremath{\Gamma} }
\newcommand{\eq}[1]{Eq.~\ref{#1}}
% ------------------------------------------------------------------



% ------------------------------------------------------------------
\begin{document}
\newcommand{\thisunit}{MATH327 Tutorial (Stirling)}
\newcommand{\moddate}{Last modified 29 Feb.~2024}
\begin{center}
  {\Large \textbf{MATH327: Statistical Physics, Spring 2024}} \\[12 pt]
  {\Large \textbf{Tutorial activity \ --- \ Stirling's formula}} \\[24 pt]
\end{center}

This activity will be introduced in our 29 February tutorial, and you can continue to work on it throughout the following week.
We'll review it during our next tutorial on 7 March.

We have already made use of \href{https://en.wikipedia.org/wiki/Stirling's_approximation}{Stirling's formula} in the following form:
\begin{align*}
  \log(N!) & = N \log N - N + \cO(\log N) \approx N \log N - N &
  \mbox{for } N & \gg 1,
\end{align*}
which implies
\begin{equation*}
  N! \approx \exp\left[N\log N - N\right] = \left(\frac{N}{e}\right)^N.
\end{equation*}
This can be made more precise:
\begin{equation}
  \label{eq:full}
  N! = \sqrt{2\pi N} \left(\frac{N}{e}\right)^N \left(1 + \frac{A}{N} + \frac{B}{N^2} + \frac{C}{N^3} + \cdots\right)
\end{equation}
with calculable coefficients $A$, $B$, $C$, etc.\footnote{\href{https://en.wikipedia.org/wiki/James_Stirling_(mathematician)}{James Stirling} computed the $\sqrt{2\pi}$ while \href{https://en.wikipedia.org/wiki/Abraham_de_Moivre}{Abraham de Moivre} derived the expansion in powers of $1 / N$.  An interesting aspect of this expansion is that it is \textbf{asymptotic} --- it has a vanishing radius of convergence but can provide precise approximations if truncated at an appropriate power.}
By performing a sequence of analyses of increasing complexity, we can build up these results.

\textbf{First analysis:} Derive the bounds
\begin{equation}
  N\log N - N < \log(N!) < N\log N
\end{equation}
for $N \gg 1$.
The second bound is the easier one.
There are multiple ways to obtain the first bound.
One pleasant approach is to consider the series expansion for $e^x$.
Together, these bounds establish
\begin{equation*}
  1 - \frac{1}{\log N} < \frac{\log(N!)}{N\log N} < 1 \qquad \implies \qquad \log(N!) \sim N\log N.
\end{equation*}

\textbf{Second analysis:} Compute the first term in \eq{eq:full}, $N! \approx \sqrt{2\pi N} \left(\frac{N}{e}\right)^N$.
This requires several steps, the first of which is to consider the \textbf{gamma function}
\begin{equation*}
  \Ga(N + 1) \equiv \int_0^{\infty} x^N e^{-x} \, dx.
\end{equation*}
Show that $\Ga(N + 1) = N!$ for integer $N \geq 0$.
In other words, derive the \href{https://en.wikipedia.org/wiki/Euler_integral}{Euler integral} (of the second kind)
\begin{equation}
  \label{eq:gamma}
  N! = \int_0^{\infty} x^N e^{-x} \, dx.
\end{equation}
Again, this can be done in multiple ways, including induction with integration by parts or by taking derivatives of
\begin{equation*}
  \int_0^{\infty} e^{-ax} \, dx = a^{-1}
\end{equation*}
and then setting $a = 1$.

\newpage
The next step in this second analysis is to approximate the gamma function as a gaussian integral.
Show that the integrand $x^N e^{-x} = \exp\left[N\log x - x\right]$ of \eq{eq:gamma} is maximized at $x = N$.

For $N \gg 1$, the integrand is sharply peaked around this maximum at $x = N$.
You can check this for yourself or take it as given.
We can therefore focus on a small region around this peak by changing variables to $y \equiv x - N$ and considering $\left|\frac{y}{N}\right| \ll 1$.
Expand the $\log x$ in the integrand, up to and including terms quadratic in $\frac{y}{N}$.
You should be left with the desired result, except for the following factor, which can be approximated by a gaussian integral (note the lower bound of integration):
\begin{equation*}
  \int_{-N}^{\infty} e^{-y^2 / (2N)} \, dy \approx \int_{-\infty}^{\infty} e^{-y^2 / (2N)} = \sqrt{2\pi N}.
\end{equation*}
The error introduced by extending the integration from $(-N, \infty)$ to  $(-\infty, \infty)$ is exponentially small and could be captured by computing the series of corrections suppressed by powers of $\frac{1}{N}$ in \eq{eq:full}.

This leads us to the \textbf{third analysis}: Compute some of the leading power-suppressed corrections in \eq{eq:full}.
That is, determine the coefficients $A$, $B$, etc.
Again, there are many ways to achieve this, including higher-order expansions of the $\log x$ considered above.
One pleasant approach is to compare $N!$ and $(N + 1)!$, now that we have derived the series prefactor $\sqrt{2\pi N} \left(\frac{N}{e}\right)^N$.

\end{document}
% ------------------------------------------------------------------
