% ------------------------------------------------------------------
\documentclass[12 pt]{article} % A4 paper set by geometry package below
\pagenumbering{arabic}
\setlength{\parindent}{10 mm}
\setlength{\parskip}{12 pt}

% Nimbus Sans font should be reasonably legible
\usepackage{helvet}
\renewcommand{\familydefault}{\sfdefault}
\usepackage[T1]{fontenc}  % Without this \textsterling produces $

% Section header spacing
\usepackage{titlesec}
\titlespacing\section{0pt}{12pt plus 4pt minus 2pt}{0pt plus 2pt minus 2pt}
\titlespacing\subsection{0pt}{12pt plus 4pt minus 2pt}{0pt plus 2pt minus 2pt}
\titlespacing\subsubsection{0pt}{12pt plus 4pt minus 2pt}{0pt plus 2pt minus 2pt}

\usepackage{amsmath}
\usepackage{amssymb}
\usepackage{graphicx}
\usepackage{verbatim}    % For comment
\usepackage[shortlabels]{enumitem}
\usepackage[paper=a4paper, marginparwidth=0 cm, marginparsep=0 cm, top=2.5 cm, bottom=2.5 cm, left=3 cm, right=3 cm, includemp]{geometry}
\usepackage[pdftex, pdfstartview={FitH}, pdfnewwindow=true, colorlinks=true, citecolor=blue, filecolor=blue, linkcolor=blue, urlcolor=blue, pdfpagemode=UseNone]{hyperref}

% Put module code and last-modified date in footer
\usepackage{fancyhdr}
\pagestyle{fancy}
\fancyhf{}
\renewcommand{\headrulewidth}{0pt}
\cfoot{{\small \thisweek}\hfill \thepage\hfill {\small \moddate}}

% Hopefully address Canvas complaints about pdf tagging and title
%\usepackage[tagged]{accessibility}
\hypersetup {
  pdfauthor={David Schaich},
  pdftitle={Statistical Physics Homework},
}
% ------------------------------------------------------------------



% ------------------------------------------------------------------
% Shortcuts
\newcommand{\cO}{\ensuremath{\mathcal O} }
\newcommand{\be}{\ensuremath{\beta} }
\newcommand{\eps}{\ensuremath{\varepsilon} }
\newcommand{\vev}[1]{\ensuremath{\left\langle #1 \right\rangle} }
\newcommand{\pderiv}[2]{\ensuremath{\frac{\partial #1}{\partial #2}} }
\newcommand{\showmarks}[1]{\rightline{\texttt{[#1 marks]}}} % \showmarks needs to follow a blank line!
% ------------------------------------------------------------------



% ------------------------------------------------------------------
\begin{document}
\newcommand{\thisweek}{MATH327 Homework 2}
\newcommand{\moddate}{Last modified 8 Apr.~2024}
\begin{center}
  {\Large \textbf{MATH327: Statistical Physics, Spring 2024}} \\[12 pt]
  {\Large \textbf{Second homework assignment}} \\[24 pt]
\end{center}

\section*{Instructions}
Complete all four questions below and submit your solutions by file upload \href{https://canvas.liverpool.ac.uk/courses/69036/assignments/263585}{on Canvas}.
Clear and neat presentations of your workings and the logic behind them will contribute to your mark.
Use of resources beyond the module materials must be explicitly referenced in your submissions.
This assignment is \textbf{due Friday, 26 April}.
Anonymous marking is turned on and I will aim to return feedback shortly after 10 May.

\vfill
The Department recently published \href{https://canvas.liverpool.ac.uk/files/10656162/}{more detailed academic integrity guidance}.
To summarize it: \\
By submitting solutions to this assessment you affirm that you have read and understood the \href{https://www.liverpool.ac.uk/media/livacuk/tqsd/code-of-practice-on-assessment/appendix_L_cop_assess.pdf}{Academic Integrity Policy} detailed in Appendix L of the Code of Practice on Assessment, and that you have successfully passed the Academic Integrity Tutorial and Quiz in the course of your studies.
You also affirm that the work you are submitting is your own and you have not commissioned production of the work from a third party or used artificial intelligence (AI) software in an unacceptable manner to generate the work.
(Generative AI software applications include, but are not limited to, ChatGPT, Bing Chat, DALL.E and Bard.)
You also affirm that you have not copied material from another person or source, nor committed plagiarism, nor fabricated, falsified or embellished data when completing the attached piece of work.
You also affirm that you have not colluded with any other student in the preparation or production of this work.
Marks achieved on this assessment remain provisional until they are ratified by the Board of Examiners in June 2024.
Please note that your submission will be analysed by Turnitin for plagiarism and an Artificial Intelligence detection tool.
% ------------------------------------------------------------------



% ------------------------------------------------------------------
\newpage
\section*{Question 1: Energy and entropy} % Could plug in specific N...
Using the canonical ensemble with inverse temperature $\be = 1 / T$, consider a system of $N \gg 1$ indistinguishable non-interacting spins in an external magnetic field with strength $H > 0$.

\begin{enumerate}[label={(\alph*)}]
  \item What are the internal energy $\vev{E}_I$ and the entropy $S_I$ as functions of $\be$, $N$ and $H$?

  \showmarks{4}

  \item Use your results to confirm the following low- and high-temperature limits:
        \begin{align*}
               \lim_{T \to 0} \vev{E}_I & = -NH &      \lim_{T \to 0} S_I & = 0           \\
          \lim_{T \to \infty} \vev{E}_I & = 0   & \lim_{T \to \infty} S_I & = \log(N + 1)
        \end{align*}

  \showmarks{6}

  \item For low but non-zero temperatures, expand both $\vev{E}_I$ and $S_I$ in terms of $\eps \equiv e^{-2\be H} \ll 1$.
        Show that the largest temperature-dependent term in the energy expansion is proportional to $\eps$ and find the constant of proportionality.
        Show that the largest temperature-dependent term in the entropy expansion is proportional to $\be\eps$ and find the constant of proportionality.

  \showmarks{8}

  \item For high but finite temperatures, expand both $\vev{E}_I$ and $S_I$ in terms of $x \equiv 2\be H \ll 1$.
        Show that the largest temperature-dependent term in the energy expansion is proportional to $x$ and find the constant of proportionality.
        Show that the largest temperature-dependent term in the entropy expansion is proportional to $x^2$ and find the constant of proportionality.

  \showmarks{8}
\end{enumerate}

\noindent The following famous results may be useful:
\begin{align*}
  \frac{1}{1 - e^{-x}} & = \frac{1}{x} + \frac{1}{2} + \frac{x}{12} - \frac{x^3}{720} + \frac{x^5}{30\,240} + \cO\!\left(x^6\right) \\
  \log\left[1 - e^{-x}\right] & = \log(x) - \frac{x}{2} + \frac{x^2}{24} - \frac{x^4}{2880} + \frac{x^6}{181\,440} + \cO\!\left(x^7\right).
\end{align*}
% ------------------------------------------------------------------



% ------------------------------------------------------------------
\newpage
\section*{Question 2: Diesel cycle} % Could plug in specific r and C...
Consider the Diesel cycle defined by the $PV$~diagram shown below, in which the `compression' stage $1 \to 2$ and the `power' stage $3 \to 4$ are both adiabatic, while the pressure is constant during the `injection/ignition' stage $2 \to 3$, and the volume is constant during the `exhaust' stage $4 \to 1$.
The compression ratio is $r \equiv V_1 / V_2 > 1$ and the cutoff ratio is $C \equiv V_3 / V_2 > 1$, with $C < r$.

\begin{center}\includegraphics[width=0.8\textwidth]{figs/Diesel.pdf}\end{center}

\begin{enumerate}[label={(\alph*)}]
  \item By computing $W_{\text{out}}$, $W_{\text{in}}$ and $Q_{\text{in}}$, show that the Diesel cycle's efficiency is
        \begin{equation*}
          \eta_D = 1 - \frac{f(C)}{r^{2 / 3}}
        \end{equation*}
        and determine the function $f(C)$ that depends only on the cutoff ratio.

  \showmarks{18}

  \item Show $f(C) > 1$ for $C > 1$.
        This will confirm a claim made in a tutorial, that the Diesel cycle is less efficient than the Otto cycle with the same compression ratio $r$.

  \showmarks{6}
\end{enumerate}
% ------------------------------------------------------------------



% ------------------------------------------------------------------
\vfill
\section*{Question 3: Particle number fluctuations}
Consider the fugacity expansion of the grand-canonical partition function (Eq.~82 in the lecture notes):
\begin{equation*}
  Z_g(T, \mu) = \sum_{N = 0}^{\infty} \xi^N \, Z_N(T).
\end{equation*}
Here $\xi = e^{\be \mu} = e^{\mu / T}$ is the fugacity and $Z_N(T)$ is the $N$-particle canonical partition function, which is independent of $\xi$.
Recall that $\Phi(T, \mu) = -T \log Z_g(T, \mu)$ is the corresponding grand-canonical potential.

\begin{enumerate}[label={(\alph*)}]
  \item Derive a relation between the average particle number $\vev{N}$ and the derivative $\displaystyle \pderiv{}{\log \xi}\Phi$.

  \showmarks{8}

  \item Derive a relation between $\vev{\left(N - \vev{N}\right)^2}$ and $\displaystyle \left(\pderiv{}{\log \xi}\right)^2 \Phi$.

  \showmarks{8}

  \item Specialize to the case of Maxwell--Boltzmann statistics, for which the fugacity expansion simplifies to $Z_g^{\text{MB}}(T, \mu) = \exp[\xi Z_1(T)]$, where $Z_1$ is the single-particle partition function.
        Use the relations you have derived to determine $\vev{N}$ and $\vev{\left(N - \vev{N}\right)^2}$ for this case, and show
        \begin{equation*}
          \frac{\sqrt{\vev{\left(N - \vev{N}\right)^2}}}{\vev{N}} = \frac{1}{\sqrt{\vev{N}}}.
        \end{equation*}
        It may be useful to note $\displaystyle \pderiv{}{\log \xi} = \xi \pderiv{}{\xi}$.

  \showmarks{8}
\end{enumerate}
% ------------------------------------------------------------------



% ------------------------------------------------------------------
\vfill
\section*{Question 4: Bosons and fermions}
Using the grand-canonical ensemble with temperature $T = 1 / \be$ and chemical potential $\mu$, consider a quantum system that has only three energy levels, with energies $E_0 = 0$, $E_1 = \eps$ and $E_2 = 2\eps$.
\begin{enumerate}[label={(\alph*)}]
  \item If the particles in this system are fermions, how many micro-states are there?

  \showmarks{2}

  \item How many micro-states would there be if the particles were bosons?

  \showmarks{2}

  \item For the fermionic case, write down every term in the grand-canonical partition function.

  \showmarks{6}

  \item For the fermionic case, what is the average internal energy $\vev{E}$?

  \showmarks{8}

  \item For the fermionic case, what are the average occupation numbers $\vev{n_0}$, $\vev{n_1}$ and $\vev{n_2}$ for the three energy levels?

  \showmarks{8}
\end{enumerate}

\noindent It may be useful to consider a system with only two energy levels as a warm-up exercise.
% ------------------------------------------------------------------



% ------------------------------------------------------------------
\end{document}
% ------------------------------------------------------------------
